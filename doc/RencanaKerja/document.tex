\documentclass[a4paper,twoside]{article}
\usepackage[T1]{fontenc}
\usepackage[bahasa]{babel}
\usepackage{graphicx}
\usepackage{graphics}
\usepackage{float}
\usepackage[cm]{fullpage}
\pagestyle{myheadings}
\usepackage{etoolbox}
\usepackage{setspace} 
\usepackage{lipsum} 
\setlength{\headsep}{30pt}
\usepackage[inner=2cm,outer=2.5cm,top=2.5cm,bottom=2cm]{geometry} %margin
% \pagestyle{empty}

\makeatletter
\renewcommand{\@maketitle} {\begin{center} {\LARGE \textbf{ \textsc{\@title}} \par} \bigskip {\large \textbf{\textsc{\@author}} }\end{center} }
\renewcommand{\thispagestyle}[1]{}
\markright{\textbf{\textsc{AIF401/AIF402 \textemdash Rencana Kerja Skripsi \textemdash Sem. Ganjil 2017/2018}}}

\onehalfspacing
 
\begin{document}

\title{\@judultopik}
\author{\nama \textendash \@npm} 

%tulis nama dan NPM anda di sini:
\newcommand{\nama}{Nancy Valentina}
\newcommand{\@npm}{2014730049}
\newcommand{\@judultopik}{Aplikasi Pratinjau 3 Dimensi Berbasis Web} % Judul/topik anda
\newcommand{\jumpemb}{1} % Jumlah pembimbing, 1 atau 2
\newcommand{\tanggal}{12/09/2017}

% Dokumen hasil template ini harus dicetak bolak-balik !!!!

\maketitle

\pagenumbering{arabic}

\section{Deskripsi}
Aplikasi pratinjau 3 dimensi merupakan sebuah perangkat lunak yang membantu penggunanya untuk meninjau kembali desain dari produk yang ingin dihasilkan secara 3 dimensi, sebelum pengguna tersebut melakukan implementasi pembuatan produk. Kelebihan dari aplikasi ini adalah pengguna dapat melakukan peninjauan dari berbagai sudut pandang untuk memaksimalkan hasil dari implementasi pembuatan produk. Aplikasi pratinjau tiga dimensi juga memungkinkan pengguna untuk merubah desain dari produk, hal ini bertujuan agar dapat membantu pengguna memutuskan desain produk yang paling sesuai. Pada dasarnya aplikasi pratinjau tiga dimensi bertujuan untuk membantu pengguna agar terhindar dari hasil pembuatan produk yang tidak sesuai dengan ekspektasi pengguna.

Penggunaan teknologi {\it web} pada aplikasi 3 dimensi dapat memudahkan pengguna untuk melakukan akses aplikasi tanpa harus melakukan instalasi pada perangkat yang digunakan. Selain itu teknologi {\it web} juga dapat digunakan pada berbagai jenis sistem operasi seperti Windows, Linux, dan Mac OS sehingga dapat mencakup banyak pengguna.

Pada skripsi ini, akan dibuat perangkat lunak yang dapat memungkinkan pengguna untuk melakukan kustomisasi ruang belajar mengajar pada lingkungan perkuliahan. Melalui perangkat lunak ini, pengguna diharapkan dapat memiliki gambaran 3 dimensi mengenai ruangan belajar mengajar dengan komposisi warna dinding dan tekstur lantai yang tepat.

Perangkat lunak akan dibuat dengan memanfaatkan WebGL dan three.js {\it library}. Sebagai studi kasus, ruangan belajar mengajar yang akan digunakan untuk melakukan simulasi aplikasi pratinjau tiga dimensi berbasis {\it web} adalah ruangan kelas Fakultas Teknologi Informasi dan Sains.

\section{Rumusan Masalah}
Berikut adalah rumusan masalah dari penulisan skripsi:
\begin{itemize}
	\item Bagaimana ruangan kelas dan objek pendukung lainnya dapat direpresentasikan dalam WebGL?
	\item Bagaimana mengkonversi hasil pratinjau ke dalam format PDF?
\end{itemize}


\section{Tujuan}
Berikut adalah tujuan dari penulisan skripsi:
\begin{itemize}
	\item Membangun aplikasi yang dapat merepresentasikan ruangan dalam WebGL.
	\item Membangun fitur konversi hasil pratinjau ke dalam format PDF.
\end{itemize}

\section{Deskripsi Perangkat Lunak}
Perangkat lunak akhir yang akan dibuat memiliki fitur minimal sebagai berikut:
\begin{itemize}
	\item Pengguna dapat mengganti tekstur lantai ruangan kelas minimal dengan 5 variasi pilihan
	\item Pengguna dapat menganti warna cat dinding ruangan kelas bagian bawah minimal dengan 5 variasi pilihan
	\item Pengguna dapat mengganti warna cat dinding ruangan kelas bagian atas minimal dengan 5 variasi pilihan
	\item Pengguna dapat melihat hasil tampilan desain ruangan kelas secara 360 derajat pada sumbu X, Y, dan Z. 
	\item Pengguna mendapatkan laporan hasil akhir dari kustomisasi dalam format PDF.
\end{itemize}

\section{Detail Pengerjaan Skripsi}
Bagian-bagian pekerjaan skripsi ini adalah sebagai berikut :
	\begin{enumerate}
		\item Mempelajari standar WebGL sebagai {\it Application Programming Interface} untuk menampilkan grafis 3 dimensi pada {\it web browser}.
		\item Mempelajari penggunaan Three.js sebagai {\it library} dari WebGL.
		\item Memodelkan ruangan belajar mengajar secara 3 dimensi.
		\item Melakukan analisis terhadap situs web yang akan dibangun.
		\item Merancang tampilan situs web yang akan dibangun.
		\item Mengimplementasikan situs web.
		\item Melakukan pengujian terhadap situs web yang telah dibangun.
		\item Menulis dokumen skripsi.
	\end{enumerate}

\section{Rencana Kerja}
\begin{center}
  \begin{tabular}{ | c | c | c | c | l |}
    \hline
    1*  & 2*(\%) & 3*(\%) & 4*(\%) &5*\\ \hline \hline
    1    & 8 & 8 &  &  \\ \hline
    2    & 8 & 8 &  &  \\ \hline
    3    & 15 & 15 &  &  \\ \hline
    4    & 6 & 6 &  &  \\ \hline
    5    & 8 &  & 8 &  \\ \hline
    6    & 30 &  & 30 &  \\ \hline
    7    & 10 &  & 10 &  \\ \hline
    8    & 15 & 3 & 12 & sebagian bab 1 dan 2, serta bagian awal analisis di S1 \\ \hline
    Total   & 100 & 40 & 60 &  \\ \hline
   \end{tabular}
\end{center}

Keterangan (*)\\
1 : Bagian pengerjaan Skripsi (nomor disesuaikan dengan detail pengerjaan di bagian 5)\\
2 : Persentase total \\
3 : Persentase yang akan diselesaikan di Skripsi 1 \\
4 : Persentase yang akan diselesaikan di Skripsi 2 \\
5 : Penjelasan singkat apa yang dilakukan di S1 (Skripsi 1) atau S2 (Skripsi 2)


\begin{center}
\break
\end{center}

\vspace{1cm}
\centering Bandung, \tanggal\\
\vspace{2cm} \nama \\ 
\vspace{1cm}


Menyetujui, \\
\ifdefstring{\jumpemb}{2}{
\vspace{1.5cm}
\begin{centering} Menyetujui,\\ \end{centering} \vspace{0.75cm}
\begin{minipage}[b]{0.45\linewidth}
% \centering Bandung, \makebox[0.5cm]{\hrulefill}/\makebox[0.5cm]{\hrulefill}/2013 \\
\vspace{2cm} Nama: \makebox[3cm]{\hrulefill}\\ Pembimbing Utama
\end{minipage} \hspace{0.5cm}
\begin{minipage}[b]{0.45\linewidth}
% \centering Bandung, \makebox[0.5cm]{\hrulefill}/\makebox[0.5cm]{\hrulefill}/2013\\
\vspace{2cm} Nama: \makebox[3cm]{\hrulefill}\\ Pembimbing Pendamping
\end{minipage}
\vspace{0.5cm}
}{
% \centering Bandung, \makebox[0.5cm]{\hrulefill}/\makebox[0.5cm]{\hrulefill}/2013\\
\vspace{2cm} Nama: \makebox[3cm]{\hrulefill}\\ Pembimbing Tunggal
}
\end{document}

