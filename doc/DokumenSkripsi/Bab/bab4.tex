\chapter{Perancangan}
\label{chapter:perancangan}
Pada bab ini dibahas mengenai perancangan perangkat lunak yang meliputi: perancangan struktur web, perancangan antarmuka, dan perancangan fitur yang diimplementasikan pada Aplikasi Pratinjau 3 Dimensi Berbasis.

\section{Perancangan Struktur Web}
\label{sec:analisisStrukturWeb}

Struktur web merupakan susunan direktori yang membangun web tersebut. Struktur web terdiri dari berbagai folder dan berkas yang telah dipisahkan berdasarkan fungsinya masing-masing. Berikut ini merupakan penjelasan masing-masing folder dan berkas yang digunakan untuk membangun web ini:

\begin{itemize}
	\item {\bf folder css}, folder ini berisi berkas dengan ekstensi css yang digunakan untuk mengatur dan memperindah tampilan web. Berkas yang ada pada folder ini hanya satu yaitu custom.css.
	\item {\bf folder js}, folder ini berisi berkas dengan ekstensi js. Terdapat berbagai file JavaScript di dalam folder ini, daftar berkasnya adalah sebagai berikut:
		\begin{itemize}
			\item {\bf Main.js}, berkas JavaScript ini berisi berbagai fungsi utama yang digunakan untuk membangun Aplikasi Pratinjau 3 Dimensi Berbasis Web.
			\item {\bf three.js}, berkas JavaScript ini berisi berbagai fungsi yang disediakan oleh pustaka Three.js. Nantinya fungsi yang terdapat di dalam berkas ini akan dipanggil oleh Main.js
			\item {\bf OrbitControls.js}, berkas JavaScript ini berisi berbagai fungsi yang juga disediakan oleh pustaka Three.js. Namun pada berkas ini hanya khusus menyediakan fungsi-fungsi yang berkaitan dengan kontrol pada kamera.
		\end{itemize}
	\item {\bf berkas index.html}, merupakan berkas {\it HyperText Markup Language} (HTML) yang membentuk web untuk aplikasi pratinjau ini.
	\item {\bf folder json}, folder ini berisi berkas dengan ekstensi JSON. Terdapat 2 berkas json di dalam folder ini yaitu sebagai berikut:
		\begin{itemize}
			\item {\bf constant.json}, berkas ini berisi berbagai informasi awal untuk diinisialisasi ke aplikasi sehingga dapat menampilkan gambaran awal dari ruangan kelas Fakultas Teknologi Informasi dan Sains.
			\item {\bf imported.json}, berkas ini berisi informasi untuk repsentasi ruangan kelas saat ujian sedang berlangsung di Fakultas Teknologi Informasi dan Sains.
		\end{itemize}
	\item {\bf folder img}, folder ini berisi berkas gambar dengan ekstensi jpg. Terdapat berkas-berkas gambar seperti pilihan tekstur warna cat dinding dan pilihan tekstur warna lantai. Berikut ini merupakan daftar berkas yang ada pada folder ini:
	\begin{itemize}
		\item textureatap.jpg, merupakan tekstur untuk bagian atap ruangan kelas.
		\item texturedinding1.jpg, merupakan pilihan tekstur pertama untuk bagian dinding ruangan kelas.
		\item texturedinding2.jpg, merupakan pilihan tekstur kedua untuk bagian dinding ruangan kelas.
		\item texturedinding3.jpg, merupakan pilihan tekstur ketiga untuk bagian dinding ruangan kelas.
		\item texturedinding4.jpg, merupakan pilihan tekstur keempat untuk bagian dinding ruangan kelas.
		\item texturedinding5.jpg, merupakan pilihan tekstur kelima untuk bagian dinding ruangan kelas.
		\item texturedinding6.jpg, merupakan pilihan tekstur keenam untuk bagian dinding ruangan kelas.
		\item texturedinding7.jpg, merupakan pilihan tekstur ketujuh untuk bagian dinding ruangan kelas.
		\item texturedinding8.jpg, merupakan pilihan tekstur kedelapan untuk bagian dinding ruangan kelas.
		\item texturelantai1.jpg, merupakan pilihan tekstur pertama untuk bagian lantai ruangan kelas.
		\item texturelantai2.jpg, merupakan pilihan tekstur kedua untuk bagian lantai ruangan kelas.
		\item texturelantai3.jpg, merupakan pilihan tekstur ketiga untuk bagian lantai ruangan kelas.
		\item texturelantai4.jpg, merupakan pilihan tekstur keempat untuk bagian lantai ruangan kelas.
		\item texturelantai5.jpg, merupakan pilhan tekstur kelima untuk bagian lantai ruangan kelas.
		\item texturelantai6.jpg, merupakan pilihan teksur keenam untuk bagian lantai ruangan kelas.
		\item texturelantai7.jpg, merupakan pilihan tekstur ketujuh untuk bagian lantai ruangan kelas.
		\item texturelantai8.jpg, merupakan pilihan tekstur kedelapan untuk bagian lantai ruangan kelas.
	\end{itemize}
	\item {\bf folder models}, folder ini berisi semua berkas yang diperlukan untuk membuat semua model pada Aplikasi Pratinjau 3 Dimensi Berbasis Web dengan ekstensi JSON (JavaScript Object Notation) dan ekstensi jpg. Berikut ini merupakan daftar berkas yang terdapat pada folder ini:
	\begin{itemize}
		\item Berkas dengan ekstensi JSON. Berkas ini berisi semua nilai informasi yang merepresentasikan suatu model seperti material yang digunakan serta koordinat titik-titik yang membentuk model tersebut. Daftar berkas dengan ekstensi JSON adalah sebagai berikut:
		\begin{itemize}
			\item ac.json, berkas yang berisi nilai informasi untuk model {\it air conditioner} pada ruangan kelas.
			\item jamdinding.json, berkas yang berisi nilai informasi untuk model jam dinding pada ruangan kelas.
			\item jendela.json, berkas yang berisi nilai informasi untuk model jendela pada ruangan kelas.
			\item kursidosen.json, berkas yang berisi nilai informasi untuk model kursi dosen pada ruangan kelas.
			\item kursimahasiswa.json, berkas yang berisi nilai informasi untuk model kursi mahasiswa pada ruangan kelas.
			\item lampu.json, berkas yang berisi nilai informasi untuk model lampu pada ruangan kelas.
			\item layar.json, berkas yang berisi nilai informasi untuk model layar proyektor pada ruangan kelas.
			\item mejadosen.json, berkas yang berisi nilai informasi untuk model meja dosen pada ruangan kelas.
			\item papantulis.json, berkas yang berisi nilai informasi untuk model papan tulis pada ruangan kelas.
			\item pintu.json, berkas yang berisi nilai informasi untuk model pintu pada ruangan kelas.
			\item proyektor.json, berkas yang berisi nilai informasi untuk model proyektor pada ruangan kelas.
		\end{itemize}
		\item Berkas dengan ekstensi jpg. Berkas ini berisi tekstur gambar yang akan dipetakan ke model dengan ekstensi JSON yang telah dibahas sebelumnya. Daftar berkas dengan ekstensi jpg pada folder ini adalah sebagai berikut:
		\begin{itemize}
			\item textureacproyektorlayar.jpg, merupakan tekstur yang akan dipetakan ke model {\it air conditioner}, proyektor, dan layar.
			\item texturejamdinding, merupakan tekstur yang akan dipetakan ke model jam dinding.
			\item texturejendela.jpg, merupakan tekstur yang akan dipetakan ke model jendela.
			\item texturekursidosen.jpg, merupakan tekstur yang akan dipetakan ke model kursi dosen.
			\item texturekursimahasiswa.jpg, merupakan tekstur yang akan dipetakan ke model kursi mahasiswa.
			\item texturelampu.jpg, merupakan tekstur yang akan dipetakan ke model lampu.
			\item texturemejadosen.jpg, merupakan tekstur yang akan dipetakan ke model meja dosen.
			\item texturepapantulis.jpg, merupakan tekstur yang akan dipetakan ke model papan tulis.
			\item texturepintu.jpg, merupakan tekstur yang akan dipetakan ke model pintu.
		\end{itemize} 
	\end{itemize}
\end{itemize}

\section{Perancangan Antarmuka}
\label{sec:perancanganAntarmuka}
Perancangan antarmuka dibuat atas dasar kebutuhan pengguna dengan Aplikasi Pratinjau 3 Dimensi Berbasis Web. Antarmuka ini digunakan sebagai media komunikasi antara pengguna dengan aplikasi web yang dibangun. Antarmuka untuk aplikasi ini terbagi menjadi dua jendela utama yaitu antarmuka masukan dan antarmuka keluaran seperti pada gambar ~\ref{fig:antarmuka1}. Berikut ini merupakan rancangan antarmuka aplikasi yang akan dibangun:
\begin{figure}[ht]
	\centering
	\includegraphics[scale=0.3]{antarmuka1}
	\caption{Rancangan antarmuka secara keseluruhan.}
	\label{fig:antarmuka1}
	\vspace{8mm}
\end{figure}

\subsection{Rancangan Antarmuka Masukan}
\label{sec:antarmukamasukan}
Melalui antarmuka ini pengguna dapat memberikan masukan untuk mengubah kondisi kelas yang sedang dipratinjau. Rancangan untuk antarmuka ini dapat dilihat pada gambar ~\ref{fig:antarmuka2}. Penjelasan untuk setiap masukan pada gambar tersebut akan dijelaskan berikut ini:
\begin{itemize}
	\item Pilihan, merupakan pilihan yang dapat diambil untuk hasil pratinjau yang telah dibuat.
		\begin{itemize}
			\item Buat ulang desain, merupakan tombol yang berfungsi untuk memperbarui halaman web sehingga masukan yang telah diberikan sebelumnya akan diabaikan dan dimulai ulang dari awal.
			\item {\it Print} hasil pratinjau, merupakan tombol yang berfungsi untuk mencetak hasil pratinjau.
		\end{itemize}
	\item Mode pratinjau, merupakan pilihan untuk posisi untuk melihat pratinjau ruangan kelas.
		\begin{itemize}
			\item Dalam kelas, merupakan tombol untuk pilihan melihat pratinjau dari dalam ruangan kelas.
			\item Luar kelas, merupakan tombol untuk pilihan melihat pratinjau dari luar ruangan kelas.
		\end{itemize}
	\item Masukan JSON, merupakan pilihan untuk memberikan masukan properti ke dalam kelas dengan ekstensi {\it JavaScript Object Notation} (JSON) sesuai format yang telah disediakan.
	\item Warna dinding, merupakan varian warna dinding yang dapat dipilih oleh pengguna untuk mengubah warna dari dinding ruangan.
	\item Warna lantai, merupakan varian warna lantai yang dapat dipilih oleh pengguna untuk mengubah warna dari lantai ruangan.
\end{itemize}
\begin{figure}[ht]
	\centering
	\includegraphics[scale=0.5]{antarmuka2}
	\caption{Rancangan antarmuka masukan.}
	\label{fig:antarmuka2}
	\vspace{8mm}
\end{figure}

\subsection{Rancangan Antarmuka Keluaran}
\label{sec:antarmukakeluaran}
Melalui antarmuka ini pengguna dapat melihat kondisi kelas yang sedang dipratinjau. Pengguna dapat melakukan rotasi untuk melihat sisi lain dari ruangan kelas yang sedang dimodelkan pada antarmuka ini. Rancangan untuk antarmuka ini dapat dilihat pada gambar \ref{fig:antarmuka3}.
\begin{figure}[ht]
	\centering
	\includegraphics[scale=0.6]{antarmuka3}
	\caption{Rancangan antarmuka keluaran.}
	\label{fig:antarmuka3}
	\vspace{8mm}
\end{figure}

\section{Perancangan Fitur yang Akan Diimplementasikan}
\label{sec:rancanganfitur}
Terdapat beberapa fitur yang akan diimplementasikan untuk Aplikasi Pratinjau 3 Dimensi. Fitur-fitur tersebut dibuat untuk mendukung kegiatan pengguna dalam merasakan proses dan hasil pratinjau yang lebih baik. Berikut ini merupakan diagram {\it use case} dalam gambar ~\ref{fig:usecase} untuk memudahkan penjelasan fitur aplikasi yang disediakan.
\begin{figure}[H]
	\centering
	\includegraphics[scale=0.6]{usecase}
	\caption{{\it Use case} diagram.}
	\label{fig:usecase}
	\vspace{8mm}
\end{figure}

\subsection{Fitur Mengganti Tekstur Warna Dinding Ruangan Kelas}
Fitur ini berguna untuk mengganti tekstur warna dinding pada model ruangan kelas agar sesuai dengan keinginan pengguna. Penjelasan lebih detail untuk fitur ini terdapat pada tabel ~\ref{table:fiturgantiteksturdinding1} dan ~\ref{table:fiturgantiteksturdinding2}.
\begin{table}[H]
	\centering
	\begin{tabular}{| m{10em} | m{30em} |} 
	\hline
	\textbf{\textit{Use case}} & \textbf{\textit{mengganti tekstur warna dinding ruangan kelas}} \\ 
	\hline
	Aktor & pengguna aplikasi  \\ 
	\hline
	Kondisi awal & aktor ingin mengganti tekstur warna dinding pada model ruangan kelas dan tekstur warna dinding pada model ruangan kelas masih tidak sesuai dengan keinginan pengguna \\ 
	\hline
	Kondisi akhir & aktor telah mengganti tekstur warna dinding pada model ruangan kelas dan tekstur warna dinding pada model ruangan kelas telah sesuai dengan keinginan pengguna \\ 
 	\hline
	Deskripsi & aktor mengganti tekstur warna dinding pada model ruangan kelas agar sesuai dengan keinginan aktor \\ 
 	\hline
	\end{tabular}
	\caption{Keterangan {\it use case} mengganti tekstur warna dinding ruangan kelas.}
	\label{table:fiturgantiteksturdinding1}
\end{table}

\begin{table}[H]
	\centering
	\begin{tabular}{| m{20em} | m{20em} |} 
	\hline
	\textbf{Pengguna} & \textbf{Aplikasi} \\ 
	\hline
	1. Memilih tekstur warna dinding. &  \\ 
	\hline
	2. Melakukan klik pada salah satu tekstur warna dinding. & 3. Eksekusi masukan dari pengguna dan melakukan perubahan pada data tekstur dinding model aplikasi. \\ 
	\hline
	 & 4. Menampilkan model ruangan kelas dengan tekstur warna dinding yang baru. \\ 
 	\hline
	\end{tabular}
	\caption{Skenario {\it use case} mengganti tekstur warna dinding ruangan kelas.}
	\label{table:fiturgantiteksturdinding2}
\end{table}

\subsection{Fitur Mengganti Tekstur Warna Lantai Ruangan Kelas}
Fitur ini berguna untuk mengganti tekstur warna lantai pada model ruangan kelas agar sesuai dengan keinginan pengguna. Penjelasan lebih detail untuk fitur ini terdapat pada tabel ~\ref{table:fiturgantiteksturlantai1} dan ~\ref{table:fiturgantiteksturlantai2}.
\begin{table}[H]
	\centering
	\begin{tabular}{| m{10em} | m{30em} |} 
	\hline
	\textbf{\textit{Use case}} & \textbf{mengganti tekstur warna lantai ruangan kelas} \\ 
	\hline
	Aktor & pengguna aplikasi  \\ 
	\hline
	Kondisi awal & aktor ingin mengganti tekstur warna lantai pada model ruangan kelas dan tekstur warna lantai pada model ruangan kelas masih tidak sesuai dengan keinginan pengguna \\ 
	\hline
	Kondisi akhir & aktor telah mengganti tekstur warna lantai pada model ruangan kelas dan tekstur warna lantai pada model ruangan kelas telah sesuai dengan keinginan pengguna \\ 
 	\hline
	Deskripsi & aktor mengganti tekstur warna lantai pada model ruangan kelas agar sesuai dengan keinginan aktor \\ 
 	\hline
	\end{tabular}
	\caption{Keterangan {\it use case} mengganti tekstur warna lantai ruangan kelas.}
	\label{table:fiturgantiteksturlantai1}
\end{table}

\begin{table}[H]
	\centering
	\begin{tabular}{| m{20em} | m{20em} |} 
	\hline
	\textbf{Pengguna} & \textbf{Aplikasi} \\ 
	\hline
	1. Memilih tekstur warna lantai. &  \\ 
	\hline
	2. Melakukan klik pada salah satu tekstur warna lantai. & 3. Eksekusi masukan dari pengguna dan melakukan perubahan pada data tekstur lantai model aplikasi. \\ 
	\hline
	 & 4. Menampilkan model ruangan kelas dengan tekstur warna lantai yang baru. \\ 
 	\hline
	\end{tabular}
	\caption{Skenario {\it use case} mengganti tekstur warna lantai ruangan kelas.}
	\label{table:fiturgantiteksturlantai2}
\end{table}

\subsection{Fitur Unggah berkas JSON untuk Mengganti Informasi Ruangan Kelas}
Fitur ini berguna untuk menambahkan dan mengganti isi informasi ruangan kelas. Informasi seperti properti ruangan kelas dapat ditambah dan diganti dengan mengunggah berkas dengan ekstensi JSON pada pilihan masukan di bagian kiri. Berkas JSON untuk beberapa mode kelas seperti saat sedang kegiatan perkuliahan, kegiatan ujian, maupun keadaan kelas yang kosong telah disediakan pada folder json. Pengguna dapat mengubah isi JSON tersebut dan menggunggah kembali untuk merubah keadaan ruangan kelas sesuai dengan yang pengguna inginkan. Contoh JSON untuk properti kelas pada saat kegiatan ujian dapat dilihat pada {\it listing} ~\ref{lst:json}. Kemudian penjelasan lebih detail untuk fitur ini terdapat pada tabel ~\ref{table:fiturunggah1} dan ~\ref{table:fiturunggah2}.
\begin{lstlisting}[caption={Contoh JSON untuk ruangan kelas pada saat ujian.}, label={lst:json},captionpos=b]
{
        "worldColor": 0xa9d9ef,
        "control": {
            "minZoom": 15,
            "maxZoom": 42
        },
        "classProperties": [
        {
            "dx": 19.5,
            "dy": 4,
            "dz": 14.2,
            "distancex": -3,
            "distancez": -3.5,
            "repeatx": 6,
            "repeaty": 5,
            "rotation": 0,
            "texture": "models/texturekursimahasiswa.jpg",
            "model": "models/kursimahasiswa.json",
            "scale": 1
        },{
            "dx": -4,
            "dy": 4,
            "dz": 14.2,
            "distancex": -3,
            "distancez": -3.5,
            "repeatx": 6,
            "repeaty": 5,
            "rotation": 0,
            "texture": "models/texturekursimahasiswa.jpg",
            "model": "models/kursimahasiswa.json",
            "scale": 1
        },
            {
                "dx": 19.5,
                "dy": 10.2,
                "dz": 17.1,
                "distancex": -3,
                "distancez": 0,
                "repeatx": 14,
                "repeaty": 1,
                "rotation": 3.14159,
                "texture": "models/texturejendela.jpg",
                "model": "models/jendela.json",
                "scale": 2
            },
            {
                "dx": 8,
                "dy": 4.7,
                "dz": -8,
                "distancex": 6,
                "distancez": 0,
                "repeatx": 2,
                "repeaty": 1,
                "rotation": Math.PI + (Math.PI/2),
                "texture": "models/texturemejadosen.jpg",
                "model": "models/mejadosen.json",
                "scale": 2
            },
            {
                "dx": -8,
                "dy": 14.5,
                "dz": -14.9,
                "distancex": 13,
                "distancez": 0,
                "repeatx": 2,
                "repeaty": 1,
                "rotation": 0,
                "texture": "models/textureacproyektorlayar.jpg",
                "model": "models/layar.json",
                "scale": 1
            },
            {
                "dx": -4,
                "dy": 14,
                "dz": 0,
                "distancex": 13,
                "distancez": 0,
                "repeatx": 2,
                "repeaty": 1,
                "rotation": Math.PI,
                "texture": "models/textureacproyektorlayar.jpg",
                "model": "models/proyektor.json",
                "scale": 1
            },
            {
                "dx": -15,
                "dy": 13,
                "dz": 14.5,
                "distancex": 15,
                "distancez": 0,
                "repeatx": 3,
                "repeaty": 1,
                "rotation": Math.PI,
                "texture": "models/textureacproyektorlayar.jpg",
                "model": "models/ac.json",
                "scale": 1
            },
            {
                "dx": -13,
                "dy": 14.9,
                "dz": -5,
                "distancex": 10,
                "distancez": 10,
                "repeatx": 3,
                "repeaty": 2,
                "rotation": 0,
                "texture": "models/texturelampu.jpg",
                "model": "models/lampu.json",
                "scale": 1
            },
            {
                "dx": 2.7,
                "dy": 14,
                "dz": -15.2,
                "distancex": 0,
                "distancez": 0,
                "repeatx": 1,
                "repeaty": 1,
                "rotation": 0,
                "texture": "models/texturejamdinding.jpg",
                "model": "models/jamdinding.json",
                "scale": 1
            },
            {
                "dx": -13,
                "dy": 7.5,
                "dz": -14.6,
                "distancex": 0,
                "distancez": 0,
                "repeatx": 1,
                "repeaty": 1,
                "rotation": 0,
                "texture": "models/texturepintu.jpg",
                "model": "models/pintu.json",
                "scale": 2
            },
            {
                "dx": 10,
                "dy": 5.5,
                "dz": -12,
                "distancex": 0,
                "distancez": 0,
                "repeatx": 1,
                "repeaty": 1,
                "rotation": 0,
                "texture": "models/texturekursidosen.jpg",
                "model": "models/kursidosen.json",
                "scale": 1.5
            }
        ],
        "room": {
            "texture": {
                "wall": [
                    "img/texturedinding1.jpg",
                    "img/texturedinding2.jpg",
                    "img/texturedinding3.jpg",
                    "img/texturedinding4.jpg",
                    "img/texturedinding5.jpg",
                    "img/texturedinding6.jpg",
                    "img/texturedinding7.jpg",
                    "img/texturedinding8.jpg"
                ],
                "floor": [
                    "img/texturelantai1.jpg",
                    "img/texturelantai2.jpg",
                    "img/texturelantai3.jpg",
                    "img/texturelantai4.jpg",
                    "img/texturelantai5.jpg",
                    "img/texturelantai6.jpg",
                    "img/texturelantai7.jpg",
                    "img/texturelantai8.jpg"
                ],
                "ceiling": "img/textureatap.jpg"
            },
            "size": {
                "length": 43,
                "width": 12,
                "height": 31
            }
        },
        "view": {
            "outside": {
                "cameraPosition": {
                    "x": 0,
                    "y": 10,
                    "z": 40
                },
                "control": {
                    "minZoom": 10,
                    "maxZoom": 42
                },
                "target": {
                    "x": 0,
                    "y": 10,
                    "z": 0
                }
            },
            "inside": {
                "cameraPosition": {
                    "x": 0,
                    "y": 10,
                    "z": 0
                },
                "control": {
                    "minZoom": 5,
                    "maxZoom": 15
                },
                "target": {
                    "x": 0,
                    "y": 10,
                    "z": 0
                }
            },
            "init": {
                "verticalField": 75,
                "nearPlane": 0.1,
                "farPlane": 100
            }
        } 
    }
\end{lstlisting}
Berikut ini merupakan penjelasan untuk setiap isi objek pada JSON tersebut:
\begin{itemize}
	\item {\it worldColor}, merupakan warna latar dari ruang tempat dilakukannya pemodelan kelas. Nilai dari objek ini harus merupakan warna dalam bentuk desimal.
	\item {\it control}, merupakan ukuran perbesaran kamera minimal dan maksimal yang dibagi menjadi objek {\it minZoom} dan {\it maxZoom}. Nilai dari objek ini harus berupa bilangan rasional.
	\item {\it classProperties}, merupakan {\it array} dari objek-objek properti yang ada di dalam ruangan kelas. Terdapat beberapa objek lagi di dalamnya dengan penjelasan sebagai berikut:
	\begin{itemize}
		\item dx, merupakan posisi pada sumbu X dari suatu properti. Nilai dari objek ini harus berupa bilangan rasional.
		\item dy, merupakan posisi pada sumbu Y dari suatu properti. Nilai dari objek ini harus berupa bilangan rasional.
		\item dz, merupakan posisi pada sumbu Z dari suatu properti. Nilai dari objek ini harus berupa bilangan rasional.
		\item distancex, merupakan jarak antara properti yang sama pada sumbu X apabila terdapat lebih dari satu properti yang sama. Nilai dari objek ini harus berupa bilangan rasional.
		\item distancez, merupakan jarak antara properti yang sama pada sumbu Z apabila terdapat lebih dari satu properti yang sama. Nilai dari objek ini harus berupa bilangan rasional. 
		\item repeatx, merupakan jumlah properti yang ingin dimunculkan pada arah sumbu X. Nilai dari objek ini harus berupa bilangan rasional.
		\item repeatz, merupakan jumlah properti yang ingin dimunculkan pada arah sumbu Z. Nilai dari objek ini harus berupa bilangan rasional.
		\item rotation, merupakan derajat rotasi dalam phi. Nilai dari objek ini merupakan kelipatan dari nilai phi.
		\item texture, merupakan alamat berkas tekstur yang ingin dipetakan pada properti tersebut. Isi dari objek ini merupakan sebuah {\it string} yang merepresentasikan alamat tekstur. 
		\item model, merupakan alamat berkas model untuk properti tersebut. Isi dari objek ini merupakan sebuah {\it string} yang merepresentasikan alamat model.
		\item scale, merupakan kelipatan perbesaran yang ingin dilakukan pada properti apabila properti dirasa kurang proposional pada ruangan kelas. Nilai dari objek ini harus berupa bilangan rasional.
	\end{itemize}
	\item room, merupakan objek yang berisi informasi ruangan kelas. Terdapat beberapa objek lagi di dalamnya dengan penjelasan sebagai berikut:
		\begin{itemize}
			\item texture, berisi pilihan tekstur untuk diaplikasikan pada ruangan kelas. Berikut ini penjelasan masing-masing objek di dalamnya:
			\begin{itemize}
				\item wall, merupakan {\it array} dari alamat-alamat berkas pilihan tekstur untuk dinding ruangan kelas. Isi dari objek ini merupakan sebuah array {\it string} yang merepresentasikan alamat tekstur.
				\item floor, merupakan {\it array} dari alamat-alamat berkas pilihan tekstur untuk lantai ruangan kelas. Isi dari objek ini merupakan sebuah array {\it string} yang merepresentasikan alamat tekstur.
				\item ceiling, merupakan alamat berkas tekstur untuk langit-langit ruangan kelas. Isi dari objek ini merupakan sebuah {\it string} yang merepresentasikan alamat tekstur.
			\end{itemize}
			\item size, merupakan ukuran panjang, lebar, dan tinggi ruangan kelas yang dibagi menjadi objek {\it length, width,} dan {\it height}. Nilai dari masing-masing objek ini harus berupa bilangan rasional.
		\end{itemize}
	\item view, merupakan objek yang berisi pengaturan perspektif kamera dengan berbagai mode. Terdapat beberapa objek untuk representasi masing-masing mode yang akan dijelaskan berikut ini:
		\begin{itemize}
			\item outside, merupakan objek untuk mode kamera saat berada di luar ruangan kelas. Berikut ini masing-masing penjelasan informasi objek yang ada pada mode ini:
			\begin{itemize}
				\item cameraPosition, merupakan posisi kamera pada setiap sumbu yang terbagi menjadi objek x, y, dan z. Nilai dari masing-masing objek ini harus berupa bilangan rasional.
				\item control, merupakan nilai minimal dan maksimal jarak perbesaran yang dapat dilakukan kamera pada mode ini sehingga terbagi menjadi objek minZoom dan maxZoom. Nilai dari masing-masing objek ini harus berupa bilangan rasional.
				\item target, merupakan titik tempat kemera mengarah sehingga dibagi menjadi objek x, y, dan z. Nilai dari masing-masing objek ini harus berupa bilangan rasional.
			\end{itemize}
			\item inside, merupakan objek untuk mode kamera saat berada di dalam ruangan kelas. Berikut ini masing-masing penjelasan informasi objek yang ada pada mode ini:
			\begin{itemize}
				\item cameraPosition, merupakan posisi kamera pada setiap sumbu yang terbagi menjadi objek x, y, dan z. Nilai dari masing-masing objek ini harus berupa bilangan rasional.
				\item control, merupakan nilai minimal dan maksimal jarak perbesaran yang dapat dilakukan kamera pada mode ini sehingga terbagi menjadi objek minZoom dan maxZoom. Nilai dari masing-masing objek ini harus berupa bilangan rasional.
				\item target, merupakan titik tempat kemera mengarah sehingga dibagi menjadi objek x, y, dan z. Nilai dari masing-masing objek ini harus berupa bilangan rasional.
			\end{itemize}
			\item init, merupakan objek untuk inisialisasi awal kamera saat aplikasi ini dibuka. Berikut ini masing-masing penjelasan informasi objek yang ada pada mode ini:
			\begin{itemize}
				\item verticalField, merupakan nilai frustum pandang vertikal untuk kamera. Nilai dari objek ini harus merupakan bilangan rasional.
				\item nearPlane, merupakan nilai frustum jarak dekat untuk kamera. Nilai dari objek ini harus merupakan bilangan rasional.
				\item farPlane, merupakan nilai frustum jarak jauh untuk kamera. Nilai dari objek ini harus merupakan bilangan rasional.
			\end{itemize}
		\end{itemize}
\end{itemize}


\begin{table}[H]
	\centering
	\begin{tabular}{| m{10em} | m{30em} |} 
	\hline
	\textbf{\textit{Use case}} & \textbf{unggah berkas JSON untuk mengganti informasi ruangan kelas.} \\ 
	\hline
	Aktor & pengguna aplikasi  \\ 
	\hline
	Kondisi awal & aktor ingin mengkustom pemodelan ruangan kelas hingga sesuai dengan keinginannya dan ruangan kelas dalam keadaan belum sesuai dengan ekspektasi aktor  \\ 
	\hline
	Kondisi akhir & aktor telah mengkustom pemodelan ruangan kelas dan ruangan kelas telah sesuai dengan ekspektasi aktor \\ 
 	\hline
	Deskripsi & aktor mengganti informasi pada model ruangan kelas dengan mengunggah berkas JSON agar sesuai dengan keinginan aktor \\ 
 	\hline
	\end{tabular}
	\caption{Keterangan {\it use case} unggah berkas JSON.}
	\label{table:fiturunggah1}
\end{table}

\begin{table}[H]
	\centering
	\begin{tabular}{| m{20em} | m{20em} |} 
	\hline
	\textbf{Pengguna} & \textbf{Aplikasi} \\ 
	\hline
	1. Memperkirakan informasi kelas apa saja yang ingin dikustom. &  \\ 
	\hline
	2. Melakukan sunting pada berkas JSON yang tersedia atau membuat berkas JSON baru dengan mengikuti format berkas JSON yang ada. &  \\ 
	\hline
	3. Menyimpan berkas JSON yang telah disunting atau dibuat baru. &  \\ 
	\hline
	4. Melakukan klik pada pemilihan unggah berkas JSON. &  \\ 
	\hline
	5. Memilih berkas JSON yang telah disunting atau dibuat baru dan mengunggahnya. & 6. Eksekusi berkas JSON yang telah diunggah pengguna. \\ 
	\hline
	 & 7. Mengganti informasi kelas sesuai dengan berkas JSON yang telah diunggah pengguna dan membangun ulang pemodelan ruangan kelas. \\ 
 	\hline
	\end{tabular}
	\caption{Skenario {\it use case} unggah berkas JSON.}
	\label{table:fiturunggah2}
\end{table}

\subsection{Fitur Mengganti Pilihan Tekstur Warna Dinding Ruangan Kelas}
Fitur ini berguna untuk mengganti pilihan tekstur warna dinding model ruangan kelas dari pilihan yang sebelumnya telah disediakan. Pengguna dapat menggunakan gambar apapun untuk dijadikan sebagai tekstur dinding ruangan kelas, namun dalam representasi gambar tersebut menjadi dinding akan mengalami sedikit penyesuaian. Penjelasan lebih detail untuk fitur ini terdapat pada tabel ~\ref{table:fiturpilihanteksturdinding1} dan ~\ref{table:fiturpilihanteksturdinding2}.
\begin{table}[H]
	\centering
	\begin{tabular}{| m{10em} | m{30em} |} 
	\hline
	\textbf{\textit{Use case}} & \textbf{mengganti pilihan tekstur warna dinding ruangan kelas} \\ 
	\hline
	Aktor & pengguna aplikasi  \\ 
	\hline
	Kondisi awal & aktor ingin mengganti atau menambahkan pilihan tekstur warna dinding yang telah disediakan sebelumnya dan pilihan warna tekstur dinding belum sesuai dengan keinginan aktor \\ 
	\hline
	Kondisi akhir & aktor telah mengganti atau menambahkan pilihan tekstur warna dinding dan pilihan tekstur warna dinding telah sesuai dengan ekspektasi aktor \\ 
 	\hline
	Deskripsi & aktor mengganti atau menambahkan pilihan tekstur warna dinding dengan mengubah isi berkas JSON dan mengunggahnya \\ 
 	\hline
	\end{tabular}
	\caption{Keterangan {\it use case} mengganti pilihan tekstur warna dinding.}
	\label{table:fiturpilihanteksturdinding1}
\end{table}

\begin{table}[H]
	\centering
	\begin{tabular}{| m{20em} | m{20em} |} 
	\hline
	\textbf{Pengguna} & \textbf{Aplikasi} \\ 
	\hline
	1. Menyiapkan tekstur warna dinding yang baru. &  \\ 
	\hline
	2. Mengubah isi berkas JSON bagian tekstur warna dinding dengan alamat tekstur warna yang baru. &  \\ 
	\hline
	3. Menyimpan berkas JSON yang telah diubah dengan tekstur warna dinding yang baru. &  \\ 
	\hline
	4. Melakukan klik pada pemilihan unggah berkas JSON. &  \\ 
	\hline
	5. Memilih berkas JSON yang telah diubah dengan tekstur warna dinding yang baru dan mengunggahnya. & 6. Eksekusi berkas JSON yang telah diunggah pengguna. \\ 
	\hline
	 & 7. Mengganti pilihan tekstur warna dinding sesuai dengan berkas JSON yang telah diunggah pengguna dan menampilkan ulang pilihan warna tekstur dinding. \\ 
 	\hline
	\end{tabular}
	\caption{Skenario {\it use case} unggah berkas JSON.}
	\label{table:fiturpilihanteksturdinding2}
\end{table}

\subsection{Fitur Mengganti Pilihan Tekstur Warna Lantai Ruangan Kelas}
Fitur ini berguna untuk mengganti pilihan tekstur warna lantai model ruangan kelas dari pilihan yang sebelumnya telah disediakan. Pengguna dapat menggunakan gambar apapun untuk dijadikan sebagai tekstur lantai ruangan kelas, namun dalam representasi gambar tersebut menjadi lantai akan mengalami sedikit penyesuaian. Penjelasan lebih detail untuk fitur ini terdapat pada tabel ~\ref{table:fiturpilihanteksturlantai1} dan ~\ref{table:fiturpilihanteksturlantai2}.
\begin{table}[H]
	\centering
	\begin{tabular}{| m{10em} | m{30em} |} 
	\hline
	\textbf{\textit{Use case}} & \textbf{mengganti pilihan tekstur warna lantai ruangan kelas} \\ 
	\hline
	Aktor & pengguna aplikasi  \\ 
	\hline
	Kondisi awal & aktor ingin mengganti atau menambahkan pilihan tekstur warna lantai yang telah disediakan sebelumnya dan pilihan warna tekstur lantai belum sesuai dengan keinginan aktor \\ 
	\hline
	Kondisi akhir & aktor telah mengganti atau menambahkan pilihan tekstur warna lantai dan pilihan tekstur warna lantai telah sesuai dengan ekspektasi aktor \\ 
 	\hline
	Deskripsi & aktor mengganti atau menambahkan pilihan tekstur warna lantai dengan mengubah isi berkas JSON dan mengunggahnya \\ 
 	\hline
	\end{tabular}
	\caption{Keterangan {\it use case} mengganti pilihan tekstur warna lantai.}
	\label{table:fiturpilihanteksturlantai1}
\end{table}

\begin{table}[H]
	\centering
	\begin{tabular}{| m{20em} | m{20em} |} 
	\hline
	\textbf{Pengguna} & \textbf{Aplikasi} \\ 
	\hline
	1. Menyiapkan tekstur warna lantai yang baru. &  \\ 
	\hline
	2. Mengubah isi berkas JSON bagian tekstur warna lantai dengan alamat tekstur warna yang baru. &  \\ 
	\hline
	3. Menyimpan berkas JSON yang telah diubah dengan tekstur warna lantai yang baru. &  \\ 
	\hline
	4. Melakukan klik pada pemilihan unggah berkas JSON. &  \\ 
	\hline
	5. Memilih berkas JSON yang telah diubah dengan tekstur warna lantai yang baru dan mengunggahnya. & 6. Eksekusi berkas JSON yang telah diunggah pengguna. \\ 
	\hline
	 & 7. Mengganti pilihan tekstur warna lantai sesuai dengan berkas JSON yang telah diunggah pengguna dan menampilkan ulang pilihan warna tekstur lantai. \\ 
 	\hline
	\end{tabular}
	\caption{Skenario {\it use case} unggah berkas JSON.}
	\label{table:fiturpilihanteksturlantai2}
\end{table}

\subsection{Fitur Menghasilkan Cetakan Ruangan Kelas}
Fitur ini memanfaatkan fasilitas cetak yang telah disediakan oleh peramban. Namun fasilitas tersebut harus didukung dengan penataan letak konten yang tepat, oleh karena itu akan diimplementasikan penataan letak agar hasil cetak hanya merupakan pemodelan ruangan kelas saja tanpa menu yang ada pada web tersebut. Penjelasan lebih detail untuk fitur ini terdapat pada tabel ~\ref{table:fiturcetak1} dan ~\ref{table:fiturcetak2}.
\begin{table}[H]
	\centering
	\begin{tabular}{| m{10em} | m{30em} |} 
	\hline
	\textbf{\textit{Use case}} & \textbf{menghasilkan cetakan ruangan kelas} \\ 
	\hline
	Aktor & pengguna aplikasi  \\ 
	\hline
	Kondisi awal & aktor ingin mencetak hasil kustomisasi pratinjau model ruangan kelas dan hasil model ruangan kelas belum tercetak \\ 
	\hline
	Kondisi akhir & aktor telah mencetak hasil kustomisasi pratinjau model ruangan kelas dan hasil model ruangan kelas telah tercetak \\ 
 	\hline
	Deskripsi & aktor mencetak hasil kustomisasi pratinjau model ruangan kelas dengan melakukan klik pada bagian cetak hasil \\ 
 	\hline
	\end{tabular}
	\caption{Keterangan {\it use case} menghasilkan cetakan ruangan kelas.}
	\label{table:fiturcetak1}
\end{table}

\begin{table}[H]
	\centering
	\begin{tabular}{| m{20em} | m{20em} |} 
	\hline
	\textbf{Pengguna} & \textbf{Aplikasi} \\ 
	\hline
	1. Menyelesaikan kustomisasi pratinjau model ruangan kelas. &  \\ 
	\hline
	2. Melakukan kllik pada bagian cetak hasil. &  3. Eksekusi masukan dari pengguna dengan menampilkan pratinjau hasil cetak \\ 
	\hline
	4. Jika tidak sesuai pengguna melakukan klik pada tombol batal dan ke tahap 5 kemudian 1. & 5. Menutup tampilan pratinjau hasil cetak. \\ 
	\hline
	6. Jika sesuai pengguna melakukan klik pada tombol cetak dan lanjut ke tahap 7. &  7. Eksekusi masukan dan mencetak hasil prantinjau ruangan kelas \\ 
 	\hline
	\end{tabular}
	\caption{Skenario {\it use case} menghasilkan cetakan ruangan kelas.}
	\label{table:fiturcetak2}
\end{table}


\section{Perancangan Pengujian Fungsional}
\label{perancanganPengujianFungsional}
Pada bagian ini akan dijelaskan bagaimana pengujian fungsional terhadap Aplikasi Pratinjau 3 Dimensi Berbasis Web dilakukan. Pada pengujian ini akan diuji apakah semua fitur telah berjalan dengan baik pada aplikasi dan memberikan hasil keluaran yang benar.

\begin{itemize}
	\item {\bf Mengganti tekstur warna dinding ruangan kelas.}
	
	Pada bagian ini akan diuji apakah aplikasi mampu mengganti warna tekstur dinding pada model ruangan kelas sesuai dengan keinginan pengguna.
	
	\item {\bf Mengganti tekstur warna lantai ruangan kelas.}
	
	Pada bagian ini akan diuji apakah aplikasi mampu mengganti warna tekstur lantai pada model ruangan kelas sesuai dengan keinginan pengguna.
	
	\item Mengunggah berkas JSON untuk mengganti informasi dari model dengan berbagai kemungkinan seperti:
		\begin{itemize}
			\item {\bf Ruangan kelas pada saat suasana ujian.}
			
			Pada bagian ini akan diuji apakah aplikasi mampu mengganti properti ruangan kelas sesuai dengan saat suasana ujian.
			
			\item {\bf Ruangan kelas pada saat tidak ada properti.}
			
			Pada bagian ini akan diuji apakah aplikasi mampu menghilangkan seluruh properti ruangan kelas.
			
			\item {\bf Ruangan kelas pada saat suasana kuliah.}
			
			Pada bagian ini akan diuji apakah aplikasi mampu mengganti properti ruangan kelas sesuai dengan saat suasana kuliah.
			
			\item {\bf Ruangan kelas pada saat tidak ada pendingin ruangan.}
			
			Pada bagian ini akan diuji apakah aplikasi mampu menghilangkan properti pendingin ruangan kelas.
			
			\item {\bf Ruangan kelas pada saat hanya ada satu kursi di tengah ruangan.}
			
			Pada bagian ini akan diuji apakah aplikasi mampu menghilangkan seluruh properti ruangan kelas dan menyisakan satu buah kursi mahasiswa di tengah-tengah kelas.
			
			\item {\bf Ruangan kelas pada saat suasana ujian dengan dua meja pengawas.}
			
			Pada bagian ini akan diuji apakah aplikasi mampu menambahkan satu buah meja pengawas di depan papan tulis.
			
			\item {\bf Ruangan kelas pada saat suasana ujian tanpa ada meja pengawas.}
			
			Pada bagian ini akan diuji apakah aplikasi mampu menghilangkan meja dosen dari ruangan kelas.
			
			\item {\bf Ruangan kelas dengan bentuk persegi.}
			
			Pada bagian ini akan diuji apakah aplikasi mampu merubah bentuk ruangan kelas menjadi persegi.
			
			\item {\bf Ruangan kelas dengan bentuk memanjang.}
			
			Pada bagian ini akan diuji apakah aplikasi mampu merubah bentuk ruangan kelas menjadi memanjang.
			
			\item {\bf Dunia di luar ruangan kelas berwarna hitam.}
			
			Pada bagian ini akan diuji apakah aplikasi mampu merubah dunia di luar ruangan kelas menjadi berwarna hitam.
			
		\end{itemize}
	\item {\bf Mengganti pilihan tekstur warna dinding untuk ruangan kelas.}
	
	Pada bagian ini akan diuji apakah aplikasi mampu menyediakan pilihan tekstur warna dinding baru untuk dipilih pengguna nantinya.
	
	\item {\bf Mengganti pilihan tekstur warna lantai untuk ruangan kelas.}
	
	Pada bagian ini akan diuji apakah aplikasi mampu menyediakan pilihan tekstur warna lantai baru untuk dipilih pengguna nantinya.
	
	\item {\bf Hasil cetakan berhasil mengambil gambar ruangan kelas.}
	
	Pada bagian ini akan diuji apakah aplikasi mampu memberikan pratinjau cetakan yang rapih hasil dari pratinjau ruangan kelas.
\end{itemize}


















