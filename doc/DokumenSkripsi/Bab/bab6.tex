\chapter{Kesimpulan dan Saran}
Bab ini membahas tentang kesimpulan berdasarkan hasil dari analisis, implementasi, dan pengujian perangkat lunak dan yang telah dibuat serta saran-saran untuk pengembangan selanjutnya.

\section{Kesimpulan}
Berdasarkan hasil dari analisis, implementasi, dan pengujian Aplikasi Prantijau 3 Dimensi Berbasis Web yang telah dibuat, telah diperoleh kesimpulan sebagai berikut:
\begin{enumerate}
	\item Berdasarkan tujuan dari skripsi ini yaitu membangun aplikasi yang dapat merepresentasikan ruangan dalam WebGL, telah dibuat Aplikasi Pratinjau 3 Dimensi Berbasis Web berdasarkan studi kasus ruangan kelas pada Fakultas Teknologi Informasi dan Sains Universitas Katolik Parahyangan dengan berbagai fitur pendukung seperti yang telah dirancang pada ~\ref{sec:rancanganfitur}. Fitur-fitur yang telah dikembangkan adalah sebagai berikut:
	\begin{itemize}
		\item Mengganti tekstur warna dinding ruangan kelas.
		\item Mengganti tekstur warna lantai ruangan kelas.
		\item Unggah berkas JavaScript Object Notation (JSON) untuk mengubah informasi ruangan kelas.
		\item Mengganti pilihan tekstur warna dinding ruangan kelas.
		\item Mengganti pilihan tekstur warna lantai ruangan kelas.
		\item Menghasilkan cetakan ruangan kelas.
	\end{itemize}
	\item Dinding dari ruangan kelas pada Fakultas Teknologi Informasi dan Sains terdiri atas dua warna. Hal tersebut tidak dapat diimplementasikan dikarenakan keterbatasan bentuk geometri yang sebelumnya telah dijelaskan pada ~\ref{sec:analisisgeometri}.
\end{enumerate}

\section{Saran}
Berdasarkan hasil pengembangan yang dilakukan, berikut ini merupakan saran-saran untuk pengembangan selanjutnya:
\begin{enumerate}
	\item Mengimplementasi fitur bentuk ruangan untuk berbagai bentuk.
	\item Mengimplementasi penyembunyian properti yang menutupi kamera saat melakukan rotasi.
	\item Mengimplementasi penyediaan antarmuka untuk kemudahan mengubah isi informasi kelas tanpa harus mengunggah berkas JSON.
\end{enumerate}